\exercisetitle{
Una línea de transmisión posee los siguientes parámetros por unidad de longitud: $L=0.3\mu H/m$, $C=450pF/m$,
$R=5\Omega /m$, y $G=0.01S/m$. Calcular la constante de propagación y la impedancia característica de esta
línea a $880MHz$. Recalcular estos parámetros en ausencia de pérdidas.}

La constante de propagación en medios con perdidas se define como:
\propconstant{ = \alpha + j \beta}
Donde sustituyendo por los valores dados en el ejercicio, $L=0.3\mu H/m$, $C=450pF/m$, $R=5\ohm /m$, y $G=0.01S/m$ obtenemos:
\begin{align*}
  \alpha &= 0.226 \\
    \beta &= 64.2
\end{align*}
Y para el cálculo de la impedancia característica:
\impedance{= 25.8 + 0.01j }
\\[0.5cm]
Para el caso sin perdidas asumiremos $R = G= 0$, por lo que la constante de propagación quedará como:
\propconstantnoloss{ = 64j}
y la impedancia característica:
\impedancenoloss{= 25.8 \Omega}
