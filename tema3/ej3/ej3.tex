\exercisetitle{
Una línea de transmisión sin pérdidas de impedancia característica $Z_0$ se termina con una impedancia de
carga de $150 \Omega$.  Si se mide una SWR en la línea de 1.6, encontrar los dos posibles valores para $Z_0$.
}
Aunque el enunciado nos dice que existen dos posible valor para $Z_0$, solo existe uno, ya que tanto la impedancia de carga, como la de la línea (sin pérdidas), son reales.
Para resolverlo empezaremos evaluando la expresión del SWR:
\swr{ = 1.6}
Donde podemos resolver para $| \Gamma_L |$,obteniendo:
\[ | \Gamma_L | = 0.23 \]
Sabemos que al ser las dos impedancias puramente reales, el valor absoluto del coeficiente de reflexión será igual a su valor real, esto se puede observar en la expresión del coeficiente de reflexión en función de la impedancia de carga y la impedancia carcterística de la línea.

\reflectionfromimpedance{ = 0.23}

De donde podemos obtener $Z_0$, el cual resulta:
\[Z_0 = 93.9 \Omega \]
