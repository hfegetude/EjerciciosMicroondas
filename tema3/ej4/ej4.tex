\exercisetitle{
Un transmisor wireless está conectado a una antena con impedancia de entrada de $80+j50\Omega$ a través de
un cable de $50\Omega$.  Si el transmisor de $50\Omega$ puede suministrar una potencia de 30W cuando se conecta a
una carga adaptada, ¿cuál es la potencia suministrada a la antena?  Repetir el cálculo suponiendo que el
transmisor tiene una impedancia de salida de $60\Omega$.
}

Nos encontramos on la situación en la que tenenemos una línea por la que circulan 30W, de los cuales el $100\%$ irán hacia la carga cuando esta este adaptada. Calcularemos que potencia irá hacia la carga en los siguientes casos:
\subsection{$Z_0 = 50\Omega$}
En este caso el coeficiente de reflexión será:
\reflectionfromimpedance{ = \frac{80+j50\Omega - 50\Omega}{80+j50\Omega + 50\Omega} = 0.418e^{j0.663} }
Y la potencia entregada:
\begin{align*}
  P_{in} &= P_{out}(1- \Gamma_L)
  P_{in} &= 30(1- 0.41)
  P_{in} &= 17.7 W
\end{align*}
\subsection{$Z_0 = 60\Omega$}
Repitiendo las cuentas:
\reflectionfromimpedance{ = \frac{80+j50\Omega - 60\Omega}{80+j50\Omega + 60\Omega} = 0.368e^{j1.533} }
Y la potencia entregada:
\begin{align*}
  P_{in} &= P_{out}(1- \Gamma_L)
  P_{in} &= 30(1- 0.368)
  P_{in} &= 18.96 W
\end{align*}
